\documentclass[12pt,letterpaper,final]{amsart}
\usepackage[utf8]{inputenc}
\usepackage{amsmath}
\usepackage{amsfonts}
\usepackage{amssymb}
\usepackage{mathtools}
\usepackage[bookmarks=true, colorlinks=true]{hyperref}
\usepackage{setspace}
% Graphics
\usepackage{graphicx}
\usepackage{epstopdf}
\DeclareGraphicsRule{.eps}{pdf}{.pdf}{`epstopdf #1}
\pdfcompresslevel=9

\setlength{\voffset}{0in}
\setlength{\hoffset}{0in}
\setlength{\oddsidemargin}{0in}
\setlength{\evensidemargin}{0in}
\setlength{\marginparwidth}{0in}
\setlength{\textwidth}{6.5in}
\setlength{\parskip}{1ex}
\doublespacing
\raggedright

\usepackage[numbers,sort&compress]{natbib}

% Front matter
\author{Joel Vincent Bernier}
\title{A general geometric model for casting diffraction}

% some macros
\newcommand{\mbm}[1]{\ensuremath{\mbox{\boldmath$#1$}}}

\newcommand{\tvecd}{\ensuremath{\mbm{\mathrm{t}}_d}}
\newcommand{\tvecs}{\ensuremath{\mbm{\mathrm{t}}_s}}
\newcommand{\tvecc}{\ensuremath{\mbm{\mathrm{t}}_c}}
\newcommand{\rmatd}{\ensuremath{\mathsf{R}_d}}
\newcommand{\rmats}{\ensuremath{\mathsf{R}_s}}
\newcommand{\rmatc}{\ensuremath{\mathsf{R}_c}}
\newcommand{\bmat}{\ensuremath{\mathsf{B}}}
\newcommand{\gvec}{\ensuremath{\mbm{\mathrm{G}}}}
\newcommand{\ghat}{\ensuremath{\hat{\mbm{\mathrm{G}}}}}
\newcommand{\bhat}{\ensuremath{\hat{\mbm{\mathrm{b}}}}}
\newcommand{\dhat}{\ensuremath{\hat{\mbm{\mathrm{d}}}}}
\newcommand{\ehat}{\ensuremath{\hat{\mbm{\mathrm{e}}}}}
\newcommand{\eye}{\ensuremath{\mbm{\mathrm{I}}}}
\newcommand{\vmat}{\ensuremath{\mbm{\mathrm{V}}}}
\newcommand{\defgrad}{\ensuremath{\mbm{\mathrm{F}}}}
\newcommand{\detpt}{\ensuremath{\mbm{\mathrm{x}}}}

\newcommand{\ghati}{\ensuremath{\hat{G}_{i}}}
\newcommand{\bhati}{\ensuremath{\hat{b}_{i}}}
\newcommand{\dhati}{\ensuremath{\hat{d}_{i}}}
\newcommand{\ghatj}{\ensuremath{\hat{G}_{j}}}
\newcommand{\bhatj}{\ensuremath{\hat{b}_{j}}}
\newcommand{\dhatj}{\ensuremath{\hat{d}_{j}}}

% points
\newcommand{\Pzero}{\ensuremath{\mathrm{P0}}}
\newcommand{\Pone}{\ensuremath{\mathrm{P1}}}
\newcommand{\Ptwo}{\ensuremath{\mathrm{P2}}}
\newcommand{\Pthree}{\ensuremath{\mathrm{P3}}}
\newcommand{\Pfour}{\ensuremath{\mathrm{P4}}}

% coordinate axes
\newcommand{\Xl}{\ensuremath{\hat{\mbm{\mathrm{X}}}_l}}
\newcommand{\Yl}{\ensuremath{\hat{\mbm{\mathrm{Y}}}_l}}
\newcommand{\Zl}{\ensuremath{\hat{\mbm{\mathrm{Z}}}_l}}
\newcommand{\labframe}{\ensuremath{\left\{\Xl\Yl\Zl\right\}}}

\newcommand{\Xd}{\ensuremath{\hat{\mbm{\mathrm{X}}}_d}}
\newcommand{\Yd}{\ensuremath{\hat{\mbm{\mathrm{Y}}}_d}}
\newcommand{\Zd}{\ensuremath{\hat{\mbm{\mathrm{Z}}}_d}}
\newcommand{\detframe}{\ensuremath{\left\{\Xd\Yd\Zd\right\}}}

\newcommand{\Xs}{\ensuremath{\hat{\mbm{\mathrm{X}}}_s}}
\newcommand{\Ys}{\ensuremath{\hat{\mbm{\mathrm{Y}}}_s}}
\newcommand{\Zs}{\ensuremath{\hat{\mbm{\mathrm{Z}}}_s}}
\newcommand{\samframe}{\ensuremath{\left\{\Xs\Ys\Zs\right\}}}

\newcommand{\Xc}{\ensuremath{\hat{\mbm{\mathrm{X}}}_c}}
\newcommand{\Yc}{\ensuremath{\hat{\mbm{\mathrm{Y}}}_c}}
\newcommand{\Zc}{\ensuremath{\hat{\mbm{\mathrm{Z}}}_c}}
\newcommand{\xtlframe}{\ensuremath{\left\{\Xc\Yc\Zc\right\}}}

% vector components
\newcommand{\labcomps}[1]{\left[#1\right]_l}
\newcommand{\detcomps}[1]{\left[#1\right]_d}
\newcommand{\samcomps}[1]{\left[#1\right]_s}
\newcommand{\xtlcomps}[1]{\left[#1\right]_c}
\newcommand{\rcpcomps}[1]{\left[#1\right]_*}

% cross-references
\newcommand{\figref}[1]{Figure~\ref{#1}}
\newcommand{\eqnref}[1]{Equation~\ref{#1}}
\newcommand{\secref}[1]{\S~\ref{#1}}

% miscellaneous
\newcommand{\ie}{{\em i.e.}}
\newcommand{\eg}{{\em e.g.}}
\newcommand{\cf}{{\em cf.}}
\newcommand{\xray}{X-ray}
\newcommand{\xrays}{X-rays}

\newcommand{\gx}{\ensuremath{\gamma_x}}
\newcommand{\gy}{\ensuremath{\gamma_y}}
\newcommand{\gz}{\ensuremath{\gamma_z}}

\newcommand{\crysdir}{\ensuremath{\mathbf{h}}}
\newcommand{\sampdir}{\ensuremath{\mathbf{s}}}
\newcommand{\hkls}{\ensuremath{hkl}}
\newcommand{\bhkls}{\ensuremath{\bar{h}\bar{k}\bar{l}}}

\newcommand{\cella}{\ensuremath{\mathbf{a}}}
\newcommand{\cellb}{\ensuremath{\mathbf{b}}}
\newcommand{\cellc}{\ensuremath{\mathbf{c}}}
\newcommand{\dalfa}{\ensuremath{\alpha}}
\newcommand{\dbeta}{\ensuremath{\beta}}
\newcommand{\dgama}{\ensuremath{\gamma}}

\newcommand{\cellParams}{\ensuremath{\left\{ a,b,c,\dalfa,\dbeta,\dgama \right\}}}
\newcommand{\cellParamsRef}{\ensuremath{\left\{ a_0,b_0,c_0,\dalfa_0,\dbeta_0,\dgama_0 \right\}}}

\newcommand{\rcella}{\ensuremath{\mathbf{a}^*}}
\newcommand{\rcellb}{\ensuremath{\mathbf{b}^*}}
\newcommand{\rcellc}{\ensuremath{\mathbf{c}^*}}
\newcommand{\ralfa}{\ensuremath{\alpha^*}}
\newcommand{\rbeta}{\ensuremath{\beta^*}}
\newcommand{\rgama}{\ensuremath{\gamma^*}}

\newcommand{\rcellParams}{\ensuremath{\left\{ a^*,b^*,c^*,\ralfa,\rbeta,\rgama \right\}}}

\newcommand{\cellVol}{\ensuremath{v}}
\newcommand{\cellVolExpr}{\ensuremath{\dotp{\cella}{\crossp{\cellb}{\cellc}}}}
\newcommand{\strain}{\mbm{\epsilon}}

% CONVENIENCE
\newcommand{\coe}{\ensuremath{\cos\eta}}
\newcommand{\sie}{\ensuremath{\sin\eta}}
\newcommand{\cob}{\ensuremath{\cos\theta}}
\newcommand{\sib}{\ensuremath{\sin\theta}}
\newcommand{\cop}{\ensuremath{\cos\phi}}
\newcommand{\sip}{\ensuremath{\sin\phi}}
\newcommand{\cow}{\ensuremath{\cos\omega}}
\newcommand{\siw}{\ensuremath{\sin\omega}}
\newcommand{\cox}{\ensuremath{\cos\chi}}
\newcommand{\six}{\ensuremath{\sin\chi}}
\newcommand{\coxsq}{\ensuremath{\cos^2\chi}}
\newcommand{\sixsq}{\ensuremath{\sin^2\chi}}
\newcommand{\sitx}{\ensuremath{\sin{2\chi}}}

\newcommand{\nZ}{\ensuremath{n_0}}
\newcommand{\nO}{\ensuremath{n_1}}
\newcommand{\nT}{\ensuremath{n_2}}
\newcommand{\nZO}{\ensuremath{n_0n_1}}
\newcommand{\nZT}{\ensuremath{n_0n_2}}
\newcommand{\nOT}{\ensuremath{n_1n_2}}
\newcommand{\hgZ}{\ensuremath{\hat{G}_0}}
\newcommand{\hgO}{\ensuremath{\hat{G}_1}}
\newcommand{\hgT}{\ensuremath{\hat{G}_2}}
\newcommand{\hbZ}{\ensuremath{\hat{b}_0}}
\newcommand{\hbO}{\ensuremath{\hat{b}_1}}
\newcommand{\hbT}{\ensuremath{\hat{b}_2}}

% MATH OPERATORS
\newcommand{\rmat}[2]{\ensuremath{\mathsf{R}\left(#1,\;#2\right)}}
\newcommand{\sub}[2]{\ensuremath{#1_#2}}
\newcommand{\dotp}[2]{\ensuremath{#1\cdot#2}}
\newcommand{\crossp}[2]{\ensuremath{#1\times#2}}
\newcommand{\modulus}[2]{\mathop{\mathrm{mod}\left(\frac{#1}{#2}\right)}}
\newcommand{\skewMat}[1]{\ensuremath{
	\left[
		\begin{array}{rrr}
			    0 & -#1_2 &  #1_1 \;\\ 
			 #1_2 &     0 & -#1_0 \;\\
			-#1_1 &  #1_0 &     0 \;
		\end{array}
	\right]}}

\newcommand{\skewOp}[1]{\ensuremath{\mathrm{skew}\;#1}}

\newcommand{\symmMat}[1]{\ensuremath{
	\left[
		\begin{array}{ccc}
			#1_0 & #1_5 & #1_4 \\ 
			     & #1_1 & #1_3 \\
			\multicolumn{2}{l}{\text{\smash{\raisebox{0.35ex}{\;Sym.}}}} & #1_2 \\
		\end{array}
	\right]}}

\newcommand{\outerProdMat}[1]{\ensuremath{
	\left[
		\begin{array}{rrr}
			#1_0^2 & #1_0#1_1 & #1_0#1_2 \\ 
			       &   #1_1^2 & #1_1#1_2 \\
			\multicolumn{2}{l}{\text{\smash{\raisebox{0.35ex}{\;\;Sym.}}}} & #1_2^2 \\
		\end{array}
	\right]}}

\newcommand{\EyeMinusOuterProdMat}[1]{\ensuremath{
	\left[
		\begin{array}{rrr}
			1 - #1_0^2 &  -#1_0#1_1 & -#1_0#1_2 \\ 
			           & 1 - #1_1^2 & -#1_1#1_2 \\
			\multicolumn{2}{l}{\text{\smash{\raisebox{0.35ex}{\;\;Sym.}}}} & 1 - #1_2^2 \\
		\end{array}
	\right]}}

% --------------------- Document ---------------------
\begin{document}
\maketitle
\begin{abstract}
A complete and general geometric model for describing diffraction
measurements is presented.  It includes provisions for describing both
mono- and poly-chromatic schemas for {\em collimated} incident beams
(\ie\ negligable beam divergence).  Detectors are modeled as planar
bodies, and their shapes and placements in the reference frame are
completely arbitrary.  The mapping that takes an admissable reciprocal
lattice vector in the crystal frame to pixel coordinates in a given
detector frame is of the form $\mathbf{A}(\mathbf{x})\cdot\mathbf{x} =
\mathbf{y}$.  The application of focus in this note is the rotation
method; this implies a monochromatic incident beam and a sample frame
that rotates about an axis that is nominally perpendicular to the
incident beam, with provision for finite deviations.
\end{abstract}

\section{Coordinate Systems}

A single-detector diffraction measurement schema is illustrated in
\figref{F:diffraction_schema}.  For genearlity we utilize four
fundamental coordinate systems:
\begin{itemize}
  \item the {\bf laboratory} frame, \labframe;
  \item the {\bf detector} frame(s), $\detframe_i$;
  \item the {\bf sample} frame, \samframe; and
  \item the {\bf crystal} frame, \xtlframe.
\end{itemize}

\subsection{Laboratory Frame}
The laboratory frame is intended to provide a global reference frame
that is stationary during the measurement.  The incident beam has a
non-neglible physical extent in the lab frame, typically possessing a
rectangular or circular cross-section.  The beam direction is
represented by the unit vector \bhat, and the centroid of its
cross-section is defined to coincide with the origin of \labframe,
labeled \Pzero.

\subsection{Detector Frame}
A typical detector element has a planar active area with a rectangular
shape.  A complete instrument may contain several independent
detectors with unique orientations and placements in the lab frame.  A
cartesian coordinate system \detframe\ is attached to each independent
detector element as follows: the origin \Pone\ is coincident with the
centroid of the active surface, and \Zd\ represents the plane normal.
The \Xd,\Yd\ directions are typically aligned with the horizontal and
vertical pixel-relative directions on each panel.  The coordinates of points 
in each independent detector coordinate system, \detframe\, are
related to coordinated in \labframe\ via a simple affine
transformation:
\begin{gather}
  \labcomps{\detpt} = \rmatd\cdot\detcomps{\detpt} + \labcomps{\tvecd}
\end{gather}
where $\tvecd = \Pone - \Pzero$ and the notation
$\labcomps{\detpt}$ indicates the components of $\detpt$ in
\labframe.  Because only planar detectors are considered, the
components of a detector-relative point are of the form
$\detcomps{\detpt}=[x\; y\; 0]$.

\subsection{Sample Frame}
The sample frame provides a basis in which to represent a uniquely
oriented and located specimen.  For the arbitrarily defined
sample-relative vector $\mathbf{S}$, coordinates in \samframe\ are
connected to \labframe\ via the affine transformation
\begin{equation}
  \labcomps{\mathbf{s}} = \rmats\cdot\samcomps{\mathbf{s}} + \labcomps{\tvecs}  
\end{equation}
where $\tvecs=\Ptwo-\Pzero$.  In the case of the rotation method, this
is frame is the oscillation frame.  Without loss of generality, the
oscillation axis is fixed to \Ys.  It may be canted with respect to
\Yl\ by the angle $\chi$\footnote{the canting angle $\chi$ is
  typically very small for an HEDM schema, although some special cases
  might require it to be set to some non-zero value.  Including this
  degree of freedom in the model also allows for it to be quantified
  via calibration.}.  The oscillation angle itself is represented by
$\omega$.  The full model has a sufficient number of degrees of
freedom to preclude the need for a third rotational degree of freedom
for \samframe.

\subsection{Crystal Frame}
The crystal frame represent a local RHON coordinate system attached to
the lattice of a single-crystal domain in the sample.  In the context
of far-field HEDM, the origin of \xtlframe, labeled \Pthree,
represents the centroid of the crystallite.  For a crystal-relative
vector $\mathbf{c}$, the components are transformed as
\begin{align}
  \samcomps{\mathbf{c}} &= \rmatc\cdot\xtlcomps{\mathbf{c}} + \samcomps{\tvecc}\\
  \labcomps{\mathbf{c}} &= \rmats\cdot\rmatc\cdot\xtlcomps{\mathbf{c}} + \rmats\cdot\samcomps{\tvecc} + \labcomps{\tvecs}
\end{align}
where $\tvecc=\Pthree-\Ptwo$.  The forumaltion of diffracted beam vectors for a given unit cell is discussed in \secref{S:diffraction}.
%%
\begin{figure}[htb]
  \centering
  \includegraphics[width=\textwidth]{new_geometry.pdf}
  \caption{A single-detector diffraction schema illustrating the four fundamental coordinate systems and the 5 generic points \Pzero-\Pfour\ used to cread the transfer function taking reciprocal lattice vector components to detector-relative components.  For completeness, the reciprocal lattice vector \gvec\ and bragg angle $2\theta$ associated with \Pfour\ are shown.  The azimuthal angle, $\eta$, of \ghat\ about \bhat\ is not shown; however, the reference azimuth is, denoted by the unit vector \ehat. }
  \label{F:diffraction_schema}
\end{figure}
%%

\newpage
\section{Diffraction}\label{S:diffraction}
\subsection{Convention for writing components of crystal lattice vectors}\label{S:conventions}
In order to insert the geometry of diffraction into the coordinate system hierarchy described above, it is necessary to etablish a convention for describing the crystal lattice. The crystal lattice itself may be
parameterized by its primitive vectors \cella, \cellb, and \cellc.
They share a common origin at a lattice site and are subject to the
following conditions:
%%
\begin{equation}
  \begin{array}{rcl}
    \|\cella\| &=& a\\
    \|\cellb\| &=& b\\
    \|\cellc\| &=& c\\
    {1 \over bc}\dotp{\cellb}{\cellc} &=& \cos\dalfa\\
    {1 \over ca}\dotp{\cellc}{\cella} &=& \cos\dbeta\\
    {1 \over ab}\dotp{\cella}{\cellb} &=& \cos\dgama
  \end{array} \label{E:lattParams}
\end{equation}
%% 
The six scalar cell parameters \cellParams\ represent a convenient parameterization of the reference unit cell, which typically correspond to an unloaded state at a reference temperature. Crystal symmetry operations (excluding triclinic) generate equivalences among cell parameters; however once a crystal is deformed, the reference symmetry is broken and all six parameters must be considered as independent.  The treatment of strained crystals in the context of the reference crystal symmetry is discussed in detail below.  For writing components of crystal-relative vectors and tensors in \xtlframe, 
a convention must be chosen to register the lattice vectors.  The
convention employed herein is consistent with that proposed by
\cite{Nye:crysBookAppB}.  Explicitly stated, $\cella\parallel\Xc$ and
$(\crossp{\cella}{\cellb})\parallel\Zc$, as depicted in
\figref{F:lattice}.

The reciprocal lattice, which forms a dual basis to the direct
lattice, is a very useful concept in diffraction.  The reciprocal lattice vectors are
defined\footnote{this is the so-called ``crystallographer's
convention'' where the prefactor of $2\pi$ is omitted.} as follows:
%%
\begin{align}
  \rcella = {1 \over v}\crossp{\cellb}{\cellc}\\
  \rcellb = {1 \over v}\crossp{\cellc}{\cella}\\
  \rcellc = {1 \over v}\crossp{\cella}{\cellb}
\end{align}
%%
where $\cellVol = \cellVolExpr$ represents the volume of the primitive
cell.  The change-of-basis matrix, \bmat, that takes vector components in the
reciprocal lattice frame to the crystal frame is formulated as
%%
\begin{align}
  \bmat &= \xtlcomps{\begin{matrix} \rcella & \rcellb & \rcellc \end{matrix}} \label{E:bmatrix}\\ 
        &= \frac{1}{\cellVol}\begin{bmatrix}
           bc\sin{\ralfa}\sin{\dbeta}\sin{\dgama} &                          0 &                                           0 \\
          -bc\sin{\ralfa}\sin{\dbeta}\cos{\dgama} & ac\sin{\ralfa}\sin{\dbeta} &                                           0 \\
          -bc(\cos{\ralfa}\sin{\dbeta}\cos{\dgama} + \cos{\dbeta}\sin{\dgama}) & ac\cos{\ralfa}\sin{\dbeta} & ab\sin{\dgama}
	\end{bmatrix}, \nonumber
\end{align}
%%
where \rcellParams\ are the reciprocal lattice parameters defined
analogously to the direct lattice parameters in \figref{F:lattice} and
\eqnref{E:lattParams}.
%%
\begin{figure}
  \centering
  \includegraphics[width=0.5\textwidth]{unitCell.pdf}
  \caption{The convention for describing the
    reference (read: unstrained) lattice has $\cella\parallel\Xc$ and
    $\rcellc\parallel\Zc$. Note that a triclinic
    primitive cell is depicted for generality.  The crystal orientation, \rmatc,
    takes components in the crystal frame, \xtlframe, to the sample
    frame, \samframe.}
  \label{F:lattice}
\end{figure}
%%

Assume that a reciprocal lattice vector, \gvec, satisfies a bragg condition.  The unit vector aligned with its associated diffracted beam, \dhat, is then
\begin{align}
\dhat &= \mathsf{R}\left(\pi, \ghat\right)\cdot\left(-\bhat\right) \nonumber\\
      &= \left(2\ghat\otimes\ghat - \eye\right)\cdot\left(-\bhat\right) \nonumber\\
      &= \left(\eye - 2\ghat\otimes\ghat\right)\cdot\bhat \label{E:dhat}\\
      & \mbox{where } \ghat\cdot\bhat = -\frac{\lambda}{2}\|\gvec\| = -\sin{\theta},  \label{E:braggCondition}\\
      & \dhat\cdot\Zd < 0,  \label{E:intersectsDetector}
\end{align}
$\ghat=\gvec / \|\gvec\|$, and \bhat\ is the unit vector aligned with the beam propagation direction.  \eqnref{E:braggCondition} is the bragg condition for \xrays\ having wavelength $\lambda$, and \eqnref{E:intersectsDetector} is the geometric condition ensuring that \dhat\ can intersect the detector plane.  Note that there is an additional condition that must be satisfied in the instrument for \Pfour\ to be observable; it must also lie within the physical extent of the detector element.  Application of this condition is, however, quite straightforward and not explicitly stated in this document.

\subsection{Representing crystal orientation and strain}
A reciprocal lattice vector, \gvec, is typically representend by its components in the reference (read: undistorted) reciprocal lattice frame, $\rcpcomps{\gvec}=[h\; k\; l]$.  The change-of-basis matrix, \bmat, that takes components in the reciprocal lattice to \xtlframe\ was given by \eqnref{E:bmatrix} above (\secref{S:conventions}).  The effects of both finite distortion and rotation of a crystal lattice with respect to \samframe\ are captured using a deformation gradient tensor, \defgrad, as presented in \cite{Bernier:2011uq}.  The tensor \defgrad\ acts on the {\em direct} lattice vectors, and takes undistorted lattice vectors in the reference crystal frame, \xtlframe, to reoriented and distorted lattice vectors in the sample frame, \samframe.  Using the polar decomposition, 
\begin{equation}
    \defgrad = \vmat \rmatc,\; \forall\; \vmat \in Sym^{3\times3},\; \rmatc \in SO^3,
\end{equation}
where \rmatc\ is the crystal orientation and \vmat\ is the ``left'' stretch tensor.  \citet{Edmiston2012} have shown that 
$\defgrad^{-T}=\vmat^{-1}\rmatc$ 
can be applied analogously to reciprocal lattice vector components written in \xtlframe.
Given the reciprocal lattice vector \gvec, with components $\rcpcomps{\gvec}=[h\; k\; l]$ in the reciprocal lattice frame, the fully distorted and orientend components in \labframe\ are then derived as
\begin{align}
\labcomps{\gvec} &= \rmats \samcomps{\gvec} \nonumber\\
                 &= \rmats \samcomps{\defgrad^{-T}} \xtlcomps{\gvec} \nonumber\\
                 &= \rmats \samcomps{\defgrad^{-T}} \bmat \rcpcomps{\gvec} \nonumber\\
        \Aboxed{ &= \rmats \samcomps{\vmat^{-1}} \rmatc \bmat \rcpcomps{\gvec} }\label{E:gvecLab},
\end{align}
The three crystal orientation degrees of freemdom are wrapped up in \rmatc, while the six that represent strain are contained in \vmat.  

Functionally, the orientation degrees of freedom can be decoupled from the strain (and position) degrees of freedom in the initial analysis.  This ``orientation indexing'' procedure essentially consists of testing discrete orientations $\mathsf{R}(\phi, \mbm{n})$, for a set of feasible \gvec\ and checking for intensity at the associated detector points \Pfour.  For the rotation method, the oscillation angle $\omega$ must be calculated for each feasible \gvec as well; this will be discussed in \secref{S:oscill}.

\section{Parametric ray-plane intersection and the transform function}\label{S:transform}
The coordinates of the points $\Pzero-\Pfour$ (see \figref{F:diffraction_schema}) in the lab frame are
\begin{align}
  \labcomps{\Pzero}  &\equiv \begin{bmatrix} 0 & 0 & 0 \end{bmatrix} \\
  \labcomps{\Pone}   &= \labcomps{\tvecd} \\
  \labcomps{\Ptwo}   &= \labcomps{\tvecs} \\
  \labcomps{\Pthree} &= \rmats\samcomps{\tvecc} + \labcomps{\tvecs} \\
  \labcomps{\Pfour}  &= \rmatd\detcomps{\detpt} + \labcomps{\tvecd}
\end{align}
Using the parametric equations for a plane (\eqnref{E:plane}) and line (\eqnref{E:line}), we may obtain a system of equations linking \dhat\ and \Pfour.
\begin{gather}
\Zd \cdot (\Pfour - \Pone) = 0 \label{E:plane}\\
\Pfour = \Pthree + u\dhat \label{E:line}
\end{gather}
Substituting for \Pfour\ and rearranging to solve for the scale parameter $u$ yields
\begin{align}
    \Zd \cdot (\Pthree - \Pone + u\dhat) &= 0 \nonumber\\
    \Zd \cdot (\Pthree - \Pone) &= -u\Zd \cdot \dhat \nonumber\\
    \implies u &= \frac{\Zd \cdot (\Pone - \Pthree)}{\Zd \cdot \dhat
    }
\end{align}
Finally, inserting this result into \eqnref{E:line} with some rearrangement yields
%\begin{equation}
%  \detcomps{\Pfour} = \rmatd^T\cdot\labcomps{ \rmats\tvecc+\tvecs-\tvecd 
%    - \frac{\left(\rmatd\Zl\right)\cdot\left(\rmats\tvecc+\tvecs-\tvecd\right)}
%           {\left(\rmatd\Zl\right)\cdot\dhat}\cdot\dhat },
%\end{equation}
%or in component form:
\begin{equation}
  \boxed{\detcomps{\Pfour} = \rmatd^T \cdot \left[ \labcomps{\Pthree - \Pone} 
    - \frac{\mathrm{dot}\left(\labcomps{\Zd}^T,\; \labcomps{\Pthree - \Pone}\right)}
           {\mathrm{dot}\left(\labcomps{\Zd}^T,\; \labcomps{\dhat}\right)}\labcomps{\dhat} \right] }, \label{E:Pfour}
\end{equation}
where $\labcomps{\dhat}$ is obtained from \eqnref{E:dhat} and \eqnref{E:gvecLab}.
% as
%\begin{align}
%  \dhati &= \left(\delta_{ij} - 2\ghati\ghatj\right)\cdot\bhatj
%\end{align}
%%
\section{Solving for the oscillation angle, $\omega$}\label{S:oscill}
For a monochromatic schema using the rotation method, the action of the oscillation in a canted configuration is represented by the rotation \rmats\, where
\begin{align}
	\rmats &= \rmat{\chi}{\Xl}\rmat{\omega}{\Yl} \label{E:rmats} \\
	&= \left[\begin{array}{rrr}1&0&0\\0&\cox&-\six\\0&\six&\cox\end{array}\right]
	   \left[\begin{array}{ccc}\cow&0&\siw\\0&1&0\\-\siw&0&\cow\end{array}\right] \nonumber\\
	&= \left[\begin{array}{ccc} 
	       \cow     &  0   &      \siw \\
	       \six\siw & \cox & -\six\cow \\
	      -\cox\siw & \six &   \cox\cow
	   \end{array}\right] . \nonumber
\end{align}
From \eqnref{E:gvecLab} we have
\begin{equation}
    \samcomps{\gvec} = \samcomps{\defgrad^{-T}}\bmat\rcpcomps{\gvec} \label{E:ghatInSamp}
\end{equation}
By writing the bragg condition (\eqnref{E:braggCondition}) in terms of lab-frame components by multiplying by the normalized components $\samcomps{\ghat}$ above by \rmats\ leads to relation of the form 
\begin{align}
    a \siw + b \cow &= c \label{E:trig}\\
                    &\equiv R\sin{\left(x+\alpha\right)} \label{E:identity}\\
                    \mbox{where } R\equiv\sqrt{a^2 + b^2} &\mbox{, and } \alpha\equiv\mathrm{atan2}(b, a),
\end{align}
and 
%%
\begin{align}
    a = \samcomps{\hgT}\labcomps{\hbZ} &+ \six\samcomps{\hgZ}\labcomps{\hbO} - \cox\samcomps{\hgZ}\labcomps{\hbT}\\
    b = \samcomps{\hgZ}\labcomps{\hbZ} &- \six\samcomps{\hgT}\labcomps{\hbO} + \cox\samcomps{\hgT}\labcomps{\hbT}\\
    c =                  -\sin{\theta} &- \cox\samcomps{\hgO}\labcomps{\hbO} - \six\samcomps{\hgO}\labcomps{\hbT}
\end{align}
%%
These yield the solutions
\begin{equation}
    \omega = \left\{
        \begin{array}{r} 
            \arcsin{ \left( \frac{c}{\sqrt{a^2 + b^2}} \right) } - \alpha \\ 
            \pi - \arcsin{ \left(\frac{c}{\sqrt{a^2 + b^2}}\right) } - \alpha \\
        \end{array}
    \right. \label{E:omegas}
\end{equation}
which are either unique (both sides of the diffraction cone), a double root (tangent to the diffraction cone -- typically not considered) or do not exist ($|\frac{c}{\sqrt{a^2 + b^2}}| > 1$, can't intersect diffraction cone).

\section{Summary}\label{S:summary}
The complete procedure for finding the detector coordinates of all reflections for a specifically oriented, strained crystal at the the position \Pthree\ consists of the following procedure:
\begin{enumerate}
\item Generate $[h\; k\; l]$ up to the relevant order (determined by geometry and wavelength of the \xrays) \\
\item Calculate the set of reciprocal lattice vector components in \samframe\ using \eqnref{E:ghatInSamp} \\
\item Calculate the set of oscillation angles, $(\omega_0,\; \omega_1)$, for each unique $[h\; k\; l]$ using \eqnref{E:omegas}\\
\item Calculate the set of unit diffraction vector components in \labframe\ using \eqnref{E:rmats}, \eqnref{E:gvecLab}, \eqnref{E:dhat} for each valid oscillation angle pair in $(\omega_0,\; \omega_1)$\\
\item Calculate $\detcomps{\Pfour}$ from \eqnref{E:Pfour}.
\end{enumerate}

 
\bibliographystyle{abbrvnat}
\bibliography{hedm}

\end{document}

